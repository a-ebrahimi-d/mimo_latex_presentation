\documentclass{beamer}

\usepackage{amsmath,mathtools,amssymb}
\usepackage{graphicx}
\usepackage{wrapfig}
\graphicspath{ {./images/} }
\usepackage{lscape}
\usepackage{rotating}
\usepackage{epstopdf}
\usepackage{sidecap}
\usepackage{xcolor}
\usepackage{multicol}
\usepackage{alltt}
\usepackage{xfrac}
\renewcommand{\ttdefault}{txtt}


%\usetheme{Boadilla}

 \usetheme[secheader]{Boadilla}
 
 \setbeamertemplate{headline}{}

      %#make sure to change this part, since it is predefined
      %\defbeamertemplate*{footline}{infolines theme}
      \setbeamertemplate{footline}
        {
      \leavevmode%
      \hbox{%
      \begin{beamercolorbox}[wd=.333333\paperwidth,ht=2.25ex,dp=1ex,center]{author in head/foot}%
        \usebeamerfont{author in head/foot}\insertshortauthor
      \end{beamercolorbox}%
      \begin{beamercolorbox}[wd=.333333\paperwidth,ht=2.25ex,dp=1ex,center]{title in head/foot}%
        \usebeamerfont{title in head/foot}\insertshorttitle
      \end{beamercolorbox}%
      \begin{beamercolorbox}[wd=.333333\paperwidth,ht=2.25ex,dp=1ex,right]{date in head/foot}%
        \usebeamerfont{date in head/foot}\insertshortdate{\hspace*{2ex}\insertframenumber{} / \inserttotalframenumber\hspace*{2ex}}

    %#turning the next line into a comment, erases the frame numbers
       % \insertframenumber{} / \inserttotalframenumber\hspace*{2ex} 

      \end{beamercolorbox}}%
      \vskip0pt%
      %\setbeamertemplate{headline}{}
    }

\definecolor{mediumturquoise}{rgb}{0.125, 0.5, 0.25}
\usecolortheme[named = mediumturquoise]{structure}
%\addtobeamertemplate{navigation symbols}{}{%
 %   \usebeamerfont{footline}%
  %  \usebeamercolor[fg]{footline}%
   % \hspace{1em}%
    %\insertframenumber/\inserttotalframenumber}

%\renewcommand{\insertnavigation}[1]{}



\title{Massive MIMO}
\author{AmirHossein Ebrahimi Derakhshan}
\institute{University of Tabriz}
\date{\today}

\begin{document}

\begin{frame}
\begin{figure}
\includegraphics[width=3cm, height=3cm]{logo.PNG}
\end{figure}
\titlepage
\end{frame}

\begin{frame}
\frametitle{Outline}
  \tableofcontents
\end{frame}

\section{Introduction}

\begin{frame}[fragile=singleslide]\frametitle{Introduction}
%\textbf{some definitions:}

\begin{itemize}
\item Cellular network :

A set of base stations (BSs) and a set of user equipments (UEs).
\end{itemize}
\begin{landscape}
\includegraphics[width=11cm, height=5.1cm]{Capture02.PNG} 
\end{landscape}


\tiny Emil Björnson, Jakob Hoydis and Luca Sanguinetti (2017), “Massive MIMO Networks: Spectral, Energy, and Hardware Efficiency”, Foundations and Trends® in Signal Processing

\end{frame}







\begin{frame}[fragile=singleslide]\frametitle{Introduction}
Each UE is connected to one of the BSs.
\begin{itemize}
\item Downlink (DL)

Signals sent from the BSs to their respective UEs 
\item Uplink (UL)

transmissions from the UEs to their respective BSs
\end{itemize}
This presentation focuses on the wireless communication links between BSs and UEs. remaining network infrastructure is assumed to function perfectly.

\begin{figure}
  \includegraphics[width=9cm, height=3.5cm]{Capture22.PNG}
  \caption{\tiny \tiny E. Bjornson, E. G. Larsson and T. L. Marzetta,"Massive mimo: ten myths and one critical question", IEEE Commun. Mag, vol. 54, pp. 114-123,2016.
}
\end{figure}
\end{frame}

\begin{frame}[fragile=singleslide]\frametitle{Introduction}
%\begin{wrapfigure}{i}{0.4\textwidth}
%   \centering
%    \includegraphics[width=3cm, height=4cm]{Capture01.PNG}
%\end{wrapfigure}
%\begin{landscape}
Several branches of wireless technologies :

\centerline{IEEE 802.11 family, 3GPP family and 3GPP2 family}


They form a heterogeneous network consisting of two main tiers:



\centerline{Coverage tier \& Hotspot tier}
 
%\hspace{1.5cm}\includegraphics[width=7.5cm, height=6cm]{Capture01.PNG} 
%\end{landscape}
\begin{figure}
  \includegraphics[width=7.5cm, height=5.5cm]{Capture01.PNG}
\end{figure}



\tiny Emil Björnson, Jakob Hoydis and Luca Sanguinetti (2017), “Massive MIMO Networks: Spectral, Energy, and Hardware Efficiency”, Foundations and Trends® in Signal Processing

\end{frame}



\begin{frame}[fragile=singleslide]\frametitle{Introduction}
\begin{enumerate}
\item Coverage tier: Outdoor cellular BSs
\begin{itemize}
\item Provide coverage, elevated base stations
\item Outdoor-to-indoor coverage: Operate $<6$GHz
\item High spectral efficiency is desired
\end{itemize}

\vspace{1.2cm}

\item Hotspot tier: Indoor BSs(mainly)
\begin{itemize}
\item High cell density, short range per cell
\item Wide bandwidths in mm-wave bands
\item Spectral efficiency less important
\end{itemize}
\end{enumerate}

\vspace{1.2cm}

\tiny Emil Björnson, Jakob Hoydis and Luca Sanguinetti (2017), “Massive MIMO Networks: Spectral, Energy, and Hardware Efficiency”, Foundations and Trends® in Signal Processing

\end{frame}


\begin{frame}[fragile=singleslide]\frametitle{The Success of Wireless Communications}

\begin{itemize}
\item More devices and data traffic every year
\begin{itemize}
\item 10\% more devices
\item 47\% more traffic (33\% more per device)
\end{itemize}
\end{itemize}
%\begin{landscape}
%\includegraphics[width=7cm, height=4cm]{Capture04.PNG}
%\caption{Data source:Ericsson Mobility Report (July, November 2017)} 
%\label{Data source:Ericsson Mobility Report (July, November 2017)}
%\end{landscape}
\begin{figure}
  \includegraphics[width=10cm, height=4.5cm]{Capture04.PNG}
\caption{Data source:Ericsson Mobility Report (July, November 2017)} 
%\label{Data source:Ericsson Mobility Report (July, November 2017)}
\end{figure}
\end{frame}



\section{Improving Cellular Networks}

\begin{frame}[fragile=singleslide]\frametitle{Improving Cellular Networks}

 Formula for Network Throughput $[bit/s/km^2]$:
\begin{equation}\label{throughput}  
\underbrace{Throughput}_{\mathbf{bit/s/km^2}}=\underbrace{Available\,spectrum}_{\mathbf{Hz}}.\underbrace{Cell\,density}_{\mathbf{Cell/km^2}}.\underbrace{Spectral\,efficiency}_{\mathbf{bit/s/Hz/Cell}}
\end{equation}


Consequently, three main ways to improve the area throughput of cellular networks:

\hspace{1cm}
\begin{enumerate}
\item Allocate more bandwidth;
\item Densify the network by deploying more BSs;
\item Improve the SE per cell.
\end{enumerate}

\vspace{1.2cm}

\tiny Emil Björnson, Jakob Hoydis and Luca Sanguinetti (2017), “Massive MIMO Networks: Spectral, Energy, and Hardware Efficiency”, Foundations and Trends® in Signal Processing

\end{frame}

%Current cellular networks utilize more than 1 GHz of bandwidth in the frequency range below 6 GHz.




\begin{frame}[t,fragile=singleslide]\frametitle{Improving Cellular Networks}
\centerline{\textit{"the 1000× data challenge"}}
 
\centerline{\textit{posed by Qualcomm for next 15–20 years}}

\vspace{0.7cm}

such a network can handle the three orders-of-magnitude increase in wireless data traffic if the annual traffic growth rate continues to be in the range of $41\% – 59\%$.

\vspace{0.7cm}

\textcolor{blue}{\textit{higher data rates, larger network capacity, higher spectral efficiency, higher energy efficiency, and better mobility}}
\vspace{0.65cm}

How can handle according to the formula in \eqref{throughput}?

\vspace{1.2cm}

\tiny Emil Björnson, Jakob Hoydis and Luca Sanguinetti (2017), “Massive MIMO Networks: Spectral, Energy, and Hardware Efficiency”, Foundations and Trends® in Signal Processing

\end{frame}




\begin{frame}[fragile=singleslide]\frametitle{Improving Cellular Networks}
\begin{itemize}
\item \textit{increase the bandwidth ?}

\end{itemize}
Currently more than 1GHz of bandwidth below 6GHz range


\centerline{$1000\times$ bandwidth}


\centerline{$\Rightarrow$ more than 1 THz of bandwidth }

            
\centerline{      \emph{This is physically impractical}}

\begin{figure}
  \includegraphics[width=11cm, height=3cm]{Capture35.PNG}
\caption{More BandWidth is not a solution} 
\end{figure}



\tiny Emil Björnson, Jakob Hoydis and Luca Sanguinetti (2017), “Massive MIMO Networks: Spectral, Energy, and Hardware Efficiency”, Foundations and Trends® in Signal Processing

\end{frame}




\begin{frame}[fragile=singleslide]\frametitle{Interference Limits the Spectral Efficiency}
\begin{itemize}
\item \textit{Densify the cellular network ?}
\end{itemize}
\begin{enumerate}
\item Limited number of locations
\item high deployment costs 
\item inter-cell interference issues
\item not suitable for mobile UEs
\end{enumerate}
\begin{figure}
  \includegraphics[width=10cm, height=4.5cm]{Capture06.PNG}
%\caption{Cell densification is not a solution} 
\end{figure}
\tiny Emil Björnson, Jakob Hoydis and Luca Sanguinetti (2017), “Massive MIMO Networks: Spectral, Energy, and Hardware Efficiency”, Foundations and Trends® in Signal Processing

\tiny YouTube channel: Wireless Future
\end{frame}


\begin{frame}[fragile=singleslide]\frametitle{How to Achieve More Uniform Coverage?}
\begin{figure}
  \includegraphics[width=11cm, height=5cm]{Capture08.PNG}
\caption{Desired: Stronger signal, same interference levels} 
\end{figure}
\tiny source: YouTube channel: Wireless Future
\end{frame}

\begin{frame}[fragile=singleslide]\frametitle{Definition of Spectral Efficiency}
The Nyquist-Shannon sampling theorem implies that the band-limited communication signal that is sent over a channel with a bandwidth of B Hz is completely determined by 2B real-valued equal-spaced samples per second

$$f < 2B$$ %Nyquist rate

\textit{Spectral efficiency:}

The SE of an encoding/decoding scheme is the average number of bits of information, per complex valued sample, that it can reliably transmit over the channel(bit/s/Hz).

$$\mathrm{ SE = \log_{2}\left(1+\frac{Transmit\, signal\, power \cdot \textcolor{green}{Pathloss}}{\textcolor{green}{interference\; power} + N_{0} \cdot \textcolor{blue}{bandwidth}}\right) }$$ %Shannon–Hartley theorem

\vspace{0.4cm}

\tiny Emil Björnson, Jakob Hoydis and Luca Sanguinetti (2017), “Massive MIMO Networks: Spectral, Energy, and Hardware Efficiency”, Foundations and Trends® in Signal Processing
\end{frame}

\section{Finding solution}

\begin{frame}[t,fragile=singleslide]\frametitle{Beamforming is the Solution!}
\begin{figure}
  \includegraphics[width=11cm, height=3.5cm]{Capture09.PNG}
 %\setbeamertemplate{caption}{\insertcaption}
 %\caption{\tiny source: YouTube channel: Wireless Future}
\end{figure}
Same transmit power \hspace{1.85cm}        More antennas


\vspace{0.35cm}
\begin{itemize}
	\begin{minipage}{0.49\linewidth}
        \item \footnotesize Color indicates path loss in dB
        \item \footnotesize M base station (BS) antennas
        \item \footnotesize Main lobe focused at user
    \end{minipage}
    \begin{minipage}{0.49\linewidth}
        \item \footnotesize Narrower beams, laser-like
        \item \scriptsize Array gain:$ 10\log_{10}$(M)dB larger at user
        \item \footnotesize Less leakage in undesired directions
   \end{minipage}
\end{itemize}

\vspace{0.65cm}
\centerline{\tiny source: YouTube channel: Wireless Future}
\end{frame}

\section{What is massive MIMO?}

\begin{frame}[t,fragile=singleslide]\frametitle{What is massive MIMO?}

Massive MIMO is a scaled up version of the conventional small scale MIMO systems

A multiuser communications solution that employs a large number of antenna elements to serve simultaneously multiple users.

\begin{figure}
  \includegraphics[width=11cm, height=3.5cm]{Capture11.PNG}
  \caption{\footnotesize Base Station \hspace{1cm}   Massive MIMO channel \hspace{1cm}    User Equipments

\vspace{0.35cm}  
  
  \tiny M. A. Albreem, M. Juntti and S. Shahabuddin, "Massive MIMO detection techniques: A survey", IEEE Commun. Surv. Tutor, 2019.
}
\end{figure}
\end{frame}

%\begin{frame}[t,fragile=singleslide]\frametitle{Beamforming is the Solution!}
%\begin{itemize}
%A highly spectrally efficient coverage tier in a cellular network characteristics:

%\item It uses SDMA to achieve a multiplexing gain

%\item It has more BS antennas than UEs per cell

%\item It operates in TDD mode to limit the CSI acquisition overhead
%\end{itemize}
%\end{frame}



\begin{frame}[t,fragile=singleslide]\frametitle{What is massive MIMO?}

New wireless access technology in 5G, in both sub-6 GHz and mmWave bands


The core technology that likely will be utilized in all future wireless technologies.
\begin{figure}
  \includegraphics[width=11cm, height=4cm]{Capture18.PNG}
  \caption{\footnotesize Illustration of downlink Massive MIMO in line-of-sight communication}
\tiny source: ma-mimo.ellintech.se
\end{figure}
\end{frame}


\begin{frame}[t,fragile=singleslide]\frametitle{What is massive MIMO?}
The key concept is to equip base stations with arrays of many antennas, which are used to serve many terminals simultaneously, in the same time-frequency resource. 


The word “massive” refer to the number of antennas and not the physical size.
%in the 2 GHz band, a half-wavelength-spaced rectangular array with 200 dual-polarized elements is about 1.5 x 0.75 meters large.
\begin{figure}
  \includegraphics[width=11cm, height=3cm]{Capture23.PNG}
  \caption{\tiny The array consists of 160 dual-polarized patch antennas. It is designed for a carrier frequency of 3.7 GHz, and the element spacing is 4 cm (half a wavelength).}
\tiny E. Bjornson, E. G. Larsson and T. L. Marzetta,"Massive mimo: ten myths and one critical question", IEEE Commun. Mag, vol. 54, pp. 114-123,2016.
\end{figure}
\tiny source: ma-mimo.ellintech.se
\end{frame}

\section{Canonical Massive MIMO network Definition}

\begin{frame}[fragile=singleslide]\frametitle{Canonical Massive MIMO network Definition:}
\begin{itemize}
\item A multicarrier cellular network with many cells
\item Operate according to a synchronous TDD protocol
\item $with \; M_{j} \geq 1 \;antennas$
\item $BS \ j \ communicates \ with \ K_{j} \ UEs \ on \ each \ time/frequency$
\item $with\; antenna-UE\; ratio\; M_{j}/K_{j} > 1$
\end{itemize}
\vspace{0.75cm}
\textit{Each BS operates individually and processes its signals using:}
\begin{itemize}
\item linear receive combining
\item linear transmit precoding
\end{itemize}

\vspace{1cm}

\tiny Emil Björnson, Jakob Hoydis and Luca Sanguinetti (2017), “Massive MIMO Networks: Spectral, Energy, and Hardware Efficiency”, Foundations and Trends® in Signal Processing
\end{frame}

\begin{frame}[t,fragile=singleslide]\frametitle{Canonical Massive MIMO network Definition:}
The canonical Massive MIMO system operates in TDD mode %where the UL and DL transmissions take place in the same frequency resource but are separated in time.
\textcolor{blue}{\textit{reasons}}:
\begin{itemize}
\item First, only the BS needs to know the channels to process the antennas coherently.
\item Second, the UL estimation overhead is proportional to the number of terminals, but independent of M
\item basic estimation theory tells us that the estimation quality (per antenna) cannot be reduced by adding more antennas
\end{itemize}
\begin{figure}
  \includegraphics[width=4cm, height=3cm]{Capture24.PNG}
\end{figure}
\tiny E. Bjornson, E. G. Larsson and T. L. Marzetta,"Massive mimo: ten myths and one critical question", IEEE Commun. Mag, vol. 54, pp. 114-123,2016.
\end{frame}


\section{Why Massive MIMO?}

\begin{frame}[fragile=singleslide]\frametitle{Why Massive MIMO?}

Massive MIMO relies on increasing the spatial multiplexing gain and the diversity gain by adding the number of antennas at the BS

with relatively simple processing of signals from all the antennas

\vspace{0.75cm}


\textit{The potential benefits of massive MIMO:}
\begin{itemize}
\item Capacity and link reliability
\item Spectral efficiency
\item Energy efficiency
\item Security enhancement and robustness improvement
\item Cost efficiency
\item Signal processing
\end{itemize}

\vspace{0.5cm}

\tiny M. A. Albreem, M. Juntti and S. Shahabuddin, "Massive MIMO detection techniques: A survey", IEEE Commun. Surv. Tutor, 2019.

\end{frame}

\begin{frame}[t,fragile=singleslide]\frametitle{Why Massive MIMO?}
\begin{itemize}
\item Capacity and link reliability:
\begin{itemize}
\item Massive MIMO increases the diversity gain, and hence, provides link robustness as it resists fading
%\scriptsize increases the diversity gain$\Rightarrow$provides link robustness$\Rightarrow$resist fading
\item The capacity increases without a bound as the number of antenna increases
%\scriptsize the capacity increases without a bound as the number of antenna
\end{itemize}



\begin{figure}
  \includegraphics[width=4.5cm, height=3cm]{Capture12.PNG}
  \includegraphics[width=4.5cm, height=3cm]{Capture50.PNG}
\end{figure}

\small $$ \mathrm{ \textcolor{red}{Multiplexed\, users} \cdot \textcolor{blue}{BW} \cdot \log_{2}\left(1+\frac{Transmit\, signal\, power \cdot \textcolor{green}{Pathloss}}{\textcolor{green}{interference power} + N_{0} \cdot \textcolor{blue}{BW}}\right) }$$

\tiny M. A. Albreem, M. Juntti and S. Shahabuddin, "Massive MIMO detection techniques: A survey", IEEE Commun. Surv. Tutor, 2019.

\tiny Inaugural talk from (SIIR) Group at Manchester Metropolitan University, UK on  "Evolving the Mobile Broadband Connectivity Towards 6G"

\tiny YouTube channel: Wireless Future
%\tiny 12th Webinar in “New Technologies for Future Generations of Wireless Systems”. Prof. Bjornson talk on “Reconfigurable Intelligent Surfaces: A Signal Processing Perspective” 
%\tiny github.com/emilbjornson/presentation_slides
\end{itemize}
\end{frame}




\begin{frame}[fragile=singleslide]\frametitle{Why Massive MIMO?}
\begin{itemize}
\item Spectral efficiency
\begin{itemize}
\item \scriptsize improves the SE of the cellular network by spatial multiplexing of a large number of UEs per cell


\item Numerous antennas


$\Rightarrow$ \scriptsize more spatial data-streams, more throughput, more multiplexing gain


$\Rightarrow$ high spectral efficiency
\end{itemize}
\end{itemize}
overall spectral efficiency in the massive MIMO can be ten times higher than in the conventional MIMO

\begin{figure}
  \includegraphics[width=5.2cm, height=2.9cm]{Capture51.PNG}
  \includegraphics[width=5.2cm, height=2.9cm]{Capture52.PNG}
\end{figure}

\tiny M. A. Albreem, M. Juntti and S. Shahabuddin, "Massive MIMO detection techniques: A survey", IEEE Commun. Surv. Tutor, 2019.

\tiny  IEEE Signal Processing Society Gujarat Section Expert talk on "MIMO communication in 5G and beyond"
%ISWCS 2018, Lisbon, Aug. 28, 2018
\end{frame}



%\begin{frame}[fragile=singleslide]\frametitle{Why Massive MIMO?}
%Energy efficiency:
%\begin{itemize}
%\item Due to coherent combining, the transmitted power is inversely proportionate to the number of transmit antennas$(\propto \frac{1}{n_{t}})$
%\item the throughput increases by increasing the number of transmit antennas and without increasing the transmit power
%\end{itemize}

%\begin{columns}[T] % align columns
%\begin{column}{.48\textwidth}


%\begin{figure}
 % \includegraphics[width=2.5cm, height=2.5cm]{Capture13.PNG}
%\end{figure}

%\begin{itemize}
%\item \footnotesize Same range with reduced power
%\begin{itemize}
%\item \tiny Increase battery lifetime in UL
%\item \tiny $Low\,power\,per\, antenna\, in\, DL\, 40W\,then4W\, per\, BS,\, 40 mW/antenna$
%\end{itemize}
%\end{itemize}


%\end{column}
%\hfill
%\begin{column}{.48\textwidth}

%\begin{figure}
%  \includegraphics[width=3.4cm, height=2.5cm]{Capture14.PNG}
%\end{figure}
%\begin{itemize}
%\item \footnotesize Use same transmit power
%\begin{itemize}
%\item \tiny Higher rates to already covered places
%\item \tiny Reach new places (e.g., indoor)
%\end{itemize}
%\end{itemize}
%\end{column}
%\end{columns}
%\end{frame}





\begin{frame}[fragile=singleslide]\frametitle{Why Massive MIMO?}
Energy efficiency:
\begin{itemize}
\item Due to coherent combining, the transmitted power is inversely proportionate to the number of transmit antennas$(\propto \frac{1}{n_{t}})$
\item the throughput increases by increasing the number of transmit antennas and without increasing the transmit power
\end{itemize}

\begin{figure}
  \includegraphics[width=2.5cm, height=2.5cm]{Capture14.PNG}
  \includegraphics[width=4cm, height=2.5cm]{Capture53.PNG}
  \includegraphics[width=3.4cm, height=2.5cm]{Capture13.PNG}
\end{figure}

Same range with reduced power \;\;\;\;\;Use same transmit power

\begin{itemize}
	\begin{minipage}{0.49\linewidth}
        \item \scriptsize Increase battery lifetime in UL
        \item \scriptsize Low power per antenna in DL
    \end{minipage}
    \begin{minipage}{0.49\linewidth}
        \item \scriptsize Reach new places (e.g., indoor)
	    \item \scriptsize Higher rates to already covered places
   \end{minipage}
\end{itemize}



\tiny M. A. Albreem, M. Juntti and S. Shahabuddin, "Massive MIMO detection techniques: A survey", IEEE Commun. Surv. Tutor, 2019.


\tiny YouTube channel: Wireless Future
\end{frame}







\begin{frame}[fragile=singleslide]\frametitle{Why Massive MIMO?}
\begin{itemize}
\item Security enhancement and robustness improvement:
\item[] \small mMIMO leads to a large number of degrees of freedom which can be used to cancel the signals from intentional jammers

\vspace{1cm}


\item Cost efficiency:
\begin{itemize}
\item \small mMIMO eliminates the need for bulky items such as coaxial cables

\item \small mMIMO uses cheap mwatts amplifier instead of a multiple expensive  HPA

\item can reduce the radiated power$\times$1000 and at the same time drastically maximize the data rates
\end{itemize}

\end{itemize}

\vspace{1cm}

\tiny M. A. Albreem, M. Juntti and S. Shahabuddin, "Massive MIMO detection techniques: A survey", IEEE Commun. Surv. Tutor, 2019.
\end{frame}



\begin{frame}[fragile=singleslide]\frametitle{Why Massive MIMO?}

Signal processing:
\begin{itemize}
\item A large number of antennas eliminates the:
\begin{itemize}
\item interference effects
\item fast fading
\item uncorrelated noise
\item thermal noise
\end{itemize}
$\Rightarrow$ simplifies the signal processing
\vspace{0.3cm}

\item channel responses from the base station to user terminals are different (mutually orthogonal, i.e., the inner products are zero).

\end{itemize}

\vspace{1.8cm}

\tiny M. A. Albreem, M. Juntti and S. Shahabuddin, "Massive MIMO detection techniques: A survey", IEEE Commun. Surv. Tutor, 2019.

\end{frame}


\end{document}
